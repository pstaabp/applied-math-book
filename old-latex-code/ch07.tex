% !TEX root = applied-math.tex


\chapter{Differential Equations on \texorpdfstring{$\mathbb{R}$}{the real number line} and the Fourier Transform}
 


\section{The Fourier Transform}

From above, we can write the Fourier Series of $f(x)$ on $[-L/2,L/2]$ as
%
\begin{align*}
f(x) & = \sum_{n=-\infty}^{\infty} c_n e^{i n \omega_0 x} \intertext{where}
c_n & = \frac{1}{L} \int_{-L/2}^{L/2} f(x) e^{-in\omega_0 x} \, dx 
\end{align*}

where $\omega_0=2\pi/L$ and $\omega_n=n\omega_0$ and let
%
\begin{align*}
\Delta \omega = \omega_{n+1}-\omega_n = \frac{2\pi}{L} 
\end{align*}

The series can be written:
%
\begin{align*}
f(x) & = \sum_{n=-\infty}^{\infty} \biggl[\frac{1}{2\pi} \int_{-L/2}^{L/2} f(x) e^{-i\omega_n x} \, dx \biggr] e^{i \omega_n x} \Delta \omega
\end{align*}
and letting $L \rightarrow \infty$ or $\Delta \omega \rightarrow 0$, we get:
%
\begin{align*}
f(x) & = \int_{-\infty}^{\infty}  \biggl[\frac{1}{2\pi} \int_{-\infty}^{\infty} f(x) e^{-i\omega_n x} \, dx \biggr] e^{i \omega x} d \omega
\end{align*}
which can be written as follows:

\begin{Boxed}
The \textbf{Fourier Transform} of $f$ is :
\begin{align*}
\hat{f}(\omega) & = \frac{1}{\sqrt{2\pi}} \int_{-\infty}^{\infty} f(x) e^{-i \omega x} \, dx 
\end{align*}
and using this, the function $f(x)$ can be written:
\begin{align*}
f(x) & = \frac{1}{\sqrt{2\pi}} \int_{-\infty}^{\infty} \hat{f}(x) e^{i \omega x} \, d\omega 
\end{align*}
Note: not all functions have a Fourier Transform. If the integral is defined (and converges), then the transform exists. 
\end{Boxed} 

(discussion of function space versus frequency space). 

\begin{example}
Find the Fourier Transform of 
%
\begin{align*}
f(x) = \begin{cases}
e^{-ax} & x \geq 0 \\
0 & \text{otherwise} 
\end{cases}
\end{align*}

\solution

\begin{align*}
\hat{f}(\omega) & =\frac{1}{\sqrt{2\pi}}\int_{-\infty}^{\infty} f(x)  e^{-i \omega x} \, dx \\
& =\frac{1}{\sqrt{2\pi}}\int_{0}^{\infty} e^{-ax} e^{-i\omega x} \, dx \\
& =\frac{1}{\sqrt{2\pi}} \int_0^{\infty} e^{-(a+i\omega) x} \, dx \\
& = \frac{1}{\sqrt{2\pi}} \frac{-1}{a+i\omega} e^{-(a + i \omega)x} \biggr\vert_0^{\infty} =\frac{1}{\sqrt{2\pi}} \frac{1}{a+i\omega} = \frac{1}{\sqrt{2\pi} }\frac{a-i\omega}{a^2+\omega^2} 
\end{align*}

\end{example}

\subsection{Even and Odd Functions and Fourier Transforms}

\begin{Boxed}
if $f(x)$ is an even function then the Fourier Transform can be written:
%
\begin{align*}
f(x) & = \int_{-\infty}^{\infty} \hat{f}(x) e^{i \omega x} \, d\omega \intertext{where}
\hat{f} (x) & = \frac{1}{2\pi} 
\end{align*}


\end{Boxed}

\begin{example}
Fourier Transform of the function
%
\begin{align*}
f(x) & = \begin{cases}
1 & | x| \leq 1 \\
0 & otherwise
\end{cases}
\end{align*}

\solution

The Fourier Transform is:
%
\begin{align*}
\hat{f}(\omega) & = \frac{1}{\sqrt{2\pi}} \int_{-\infty}^{\infty} f(x) e^{-i\omega x} \, dx \\
& = \frac{1}{\sqrt{2\pi}} \int_{-1}^{1} e^{-i\omega x} \, dx \\
& = \frac{1}{\sqrt{2\pi}} \frac{1}{-i \omega}  e^{-i\omega x} \biggr\vert_{-1}^1  \\
& = \frac{1}{\sqrt{2\pi}} \biggl( \frac{e^{-i\omega}}{-i \omega } + \frac{e^{i\omega}}{i \omega} \biggr) \\
& = \frac{1}{\sqrt{2\pi}} \frac{2 \sin \omega}{\omega}  
\end{align*}

Also, 
%
\begin{align*}
f(x) & = \frac{1}{\sqrt{2\pi}} \int_{-\infty}^{\infty} \hat{f}(\omega) e^{i \omega x} \, d\omega \\
& = \frac{1}{2\pi} \int_{-\infty}^{\infty} \frac{2 \sin \omega}{\omega} e^{i \omega x} \, d \omega 
\end{align*}

Since $f(0)=1$ (from the original definition of the function), this means that
%
\begin{align*}
1 & = \frac{1}{\pi} \int_{-\infty}^{\infty}\frac{\sin \omega}{\omega} \, d\omega  \intertext{or}
\pi & = \int_{-\infty}^{\infty}\frac{\sin \omega}{\omega} \, d\omega 
\end{align*}
since $\frac{\sin \omega}{\omega}$ is an even function we can write:
%
\begin{align*}
\frac{\pi}{2} & = \int_{0}^{\infty} \frac{\sin \omega}{\omega} \, d\omega
\end{align*}

An important function is that on the right side 
%
\begin{align*}
\mbox{Si}(u) & = \int_{0}^u \frac{\sin \omega}{\omega} \, d\omega 
\end{align*}



\end{example}


\subsection{Physical Interpretation of the Fourier Transform}

The Fourier Transform of a function $f(x)$ is a function that lists all of its frequencies, often called the \textbf{spectral representation}.  The name come from understanding light in terms of its frequencies (colors) and many analyses in physics and chemistry are easier to do in term of the spectrum 

\begin{definition}
The \textbf{total energy} is 
%
\begin{align*}
\int_{-\infty}^{\infty} | \hat{f}(\omega)|^2 \, d\omega
\end{align*}
\end{definition}


\subsection{Properties of the Fourier Transform}

\begin{theorem}[Linearity]
The Fourier Transform is \emph{linear}; that is, for any pair of functions $f(x)$ and $g(x)$ whose Fourier transforms exist and any constants $a$ and $b$, then
%
\begin{align*}
{\cal F}(af+bg) &  = a {\cal F}(f) + b {\cal F}(g)  
\end{align*}
\end{theorem}

\begin{proof}
\begin{align*}
{\cal F}(af+bg) & = \frac{1}{\sqrt{2\pi}} \int_{-\infty}^{\infty} (a f(x) + b g(x)) \, dx \\
& = a \frac{1}{\sqrt{2\pi}} \int_{-\infty}^{\infty} f(x) \, dx + b \frac{1}{\sqrt{2\pi}} \int_{-\infty}^{\infty} g(x) \, dx \\
& = a {\cal F}(f) + b {\cal F}(g) 
\end{align*}
\end{proof}


\begin{theorem}[Derivatives]
Let $f(x)$ be a continuous on the $x$-axis and $f(x) \rightarrow 0$ as $|x| \rightarrow \infty$.  Furthermore, let $f'(x)$ be absolutely integrable on the $x$-axis.  Then 
\begin{align*}
{\cal F}\bigl\{f'(x) \bigr\} & = i \omega {\cal F} \bigl\{ f(x) \bigr\} 
\end{align*}

\end{theorem}
\begin{proof}
\begin{align*}
{\cal F} \bigl\{ f'(x) \bigr\} & = \frac{1}{\sqrt{2\pi}} \int_{-\infty}^{\infty} f'(x) e^{-i \omega x} \, dx  \intertext{integrate by parts} 
& = \frac{1}{\sqrt{2\pi}} \biggl( f(x) e^{-i \omega x} \biggr\vert_{-\infty}^{\infty} - (-i\omega) \int_{-\infty}^{\infty} f(x) e^{-i \omega x} \, dx \biggr) \intertext{since $f(x) \rightarrow 0$ as $|x|\rightarrow \infty$} 
& = \frac{1}{\sqrt{2\pi}} i\omega {\cal F} \bigl\{ f(x) \bigr\} 
\end{align*}
\end{proof}




\section{Solving the Heat Equation using Fourier Transform}

Now we solve:
%
\begin{align*}
\frac{1}{\kappa}\frac{\partial u}{\partial t} & = \frac{\partial^2 u}{\partial {x}^2} 
\end{align*}
if $-\infty < x < \infty$ and $u(x,0)=f(x)$.  

Let $\hat{u}(\omega,t)$ be the Fourier Transform of $u(x,t)$.  Taking the Fourier Transform of the PDE above:
%
\begin{align*}
\frac{1}{\kappa} \frac{\partial \hat{u}}{\partial t} & = \frac{\partial^2 \hat{u}}{\partial {x}^2} 
\end{align*}
and using the derivative of the Fourier Transform:
\begin{align*}
\frac{1}{\kappa} \frac{\partial U}{\partial t} & = (i\omega)^2 \hat{u} = -\omega^2 \hat{u} 
\end{align*}
and solving this we get:
%
\begin{align*}
\hat{u}(\omega,t) & = A(\omega) e^{-\kappa \omega^2 t}   
\end{align*}

If we take the Fourier Transform of the initial condition:
%
\begin{align*}
\hat{u}(\omega,0) & = {\cal F}\{f(x)\} = \hat{f}(\omega)
\end{align*}

and then evaluate the transformed solution to get:
%
\begin{align*}
\hat{u}(\omega,0) & = A(\omega) \intertext{so}
\hat{u}(\omega,t) & = \hat{f}(\omega) e^{-\kappa \omega^2 t} 
\end{align*}
and lastly, we take the inverse transform:
%
\begin{align*}
u(x,t) & = \frac{1}{\sqrt{2\pi}} \int_{-\infty}^{\infty} \hat{f}(\omega) e^{-\kappa \omega^2 t}  e^{i\omega x} \, d\omega 
\end{align*}


\begin{example}
Solve the Heat equation on the real line if the initial condition is
%
\begin{align*}
f(x) & = \begin{cases}
1 & |x| \leq 1 \\
0 & \text{otherwise}
\end{cases}
\end{align*}

\solution

From above
%
\begin{align*}
\hat{f}(\omega) & = \frac{1}{\sqrt{2\pi}} \frac{\sin \omega}{\omega}
\end{align*}
so the solution is:
%
\begin{align*}
u(x,t) & = \frac{1}{\sqrt{2\pi}}\int_{-\infty}^{\infty}  \frac{1}{\sqrt{2\pi}} \frac{\sin \omega}{\omega} e^{-\kappa \omega^2 t + i \omega x} \, d\omega 
\end{align*}

\end{example}






